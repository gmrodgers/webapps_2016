\documentclass{article}
\usepackage[utf8]{inputenc}
\usepackage[export]{adjustbox}
\usepackage{subfigure}
\usepackage{url}
\usepackage{graphicx}
\graphicspath {
	{images/}
}

\begin{document}

\title{Design Research and Insights}
\author{Cardspark - Group 26}
\date{\today}
\maketitle 

We had the idea of a revision app due to our previous difficulties in revision so we choose to research this idea with our peers.

\begin{center}
	\vspace{1mm}
	\includegraphics[scale=0.14]{form.png}
	\vspace{1mm}
\end{center}

Using the Google form above, we surveyed our friends, their friends and then their friends as part of our market research.  This meant we didn't just get opinions from Imperial College, but from all parts of the UK.  It gave us the following results:

\begin{figure}[ht]
	\centering
	\begin{subfigure}{}
	  \centering
			\includegraphics[width=5.85cm, height=5.8cm]{images_colours.png}
	\end{subfigure}%
	\begin{subfigure}{}
	  \centering
			\includegraphics[width=5.85cm, height=5.8cm]{mobile_audio.png}
	\end{subfigure}
\end{figure}

\newpage
More importantly it gave the users a chance to say what they have problems with during revision.

\begin{center}
	\vspace{1mm}
	\includegraphics[scale=0.5]{feedback.png}
	\vspace{1mm}
\end{center}

We then took these general problems and asked some people around us to expand on this and what the core problems they had were.\\

The prevailing points from our surveys and questions were as follows:
\begin{itemize}
\item
	There are not enough opportunities to collaborate during the revision process
\item
	The notes given to our main audience, students, were often too large and difficult for them to consume in short sittings
\item
	That testing themselves was an important part of their revision and learning process
\item
	The use of images, colours and other visual cues were seen as helpful by the fast majority of the students
\end{itemize}

We then wanted to investigate this further and how much collaboration and flashcards could help form a solution to their problems.  We read papers on group work saying that groups "can save time and requires a shared workload" \cite{groupwork1} and "people remember group discussions better" \cite{groupwork2}.

\begin{thebibliography}{2}
	\bibitem{groupwork1} 
		\url{https://sydney.edu.au/education_social_work/groupwork/docs/BenefitsOfGW.pdf}
	\bibitem{groupwork2} 
		\url{http://uncw.edu/cte/et/articles/Vol11_2/Burke.pdf}
\end{thebibliography}

\end{document}